\documentclass[a4paper, titlepage, 12pt]{article}
\usepackage[utf8]{inputenc}
\usepackage[russian]{babel}
\usepackage{mathtools}
\usepackage{fixltx2e}
\usepackage{stmaryrd}
\usepackage[round, sort]{natbib}
\usepackage{gb4e}
\noautomath

% COMPILE WITH: reset && pdflatex zavgorodny_diplom.tex && bibtex zavgorodny_diplom && pdflatex zavgorodny_diplom.tex && pdflatex zavgorodny_diplom.tex

\title{Семантика и дистрибуция фокусных частиц в русском языке}

\author{Андрей Завгородний}

\renewcommand*\contentsname{Summary}

\begin{document}

\begin{titlepage}

\newcommand{\HRule}{\rule{\linewidth}{0.5mm}} % Defines a new command for the horizontal lines, change thickness here

\center % Center everything on the page
 
%----------------------------------------------------------------------------------------
%   HEADING SECTIONS
%----------------------------------------------------------------------------------------

\textsc{\large Московский государственный университет имени \\ М.В. Ломоносова}\\[1.5cm] % Name of your university/college
\textsc{\large Филологический факультет}\\[0.5cm] % Major heading such as course name
\textsc{\large Отделение фундаментальной и прикладной лингвистики}\\[0.5cm] % Minor heading such as course title

~\\[2.0cm]

%----------------------------------------------------------------------------------------
%   TITLE SECTION
%----------------------------------------------------------------------------------------


{ \huge Семантика и дистрибуция фокусных частиц в русском языке}\\[0.4cm] % Title of your document
 
%----------------------------------------------------------------------------------------
%   AUTHOR SECTION
%----------------------------------------------------------------------------------------

~\\[3.0cm]

\begin{minipage}{0.4\textwidth}
\begin{flushleft} \large
Дипломная работа студента II курса магистратуры Андрея Олеговича Завгороднего \\
\end{flushleft}
\end{minipage}
~
\begin{minipage}{0.4\textwidth}
\begin{flushright} \large
Научный руководитель д.ф.н., проф. \\  Сергей Георгиевич Татевосов \\
\end{flushright}
\end{minipage}\\[5cm]

% If you don't want a supervisor, uncomment the two lines below and remove the section above
%\Large \emph{Author:}\\
%John \textsc{Smith}\\[3cm] % Your name

%----------------------------------------------------------------------------------------
%   DATE SECTION
%----------------------------------------------------------------------------------------

{\large Москва, 2017} % Date, change the \today to a set date if you want to be precise

%----------------------------------------------------------------------------------------
%   LOGO SECTION
%----------------------------------------------------------------------------------------

%\includegraphics{Logo}\\[1cm] % Include a department/university logo - this will require the graphicx package
 
%----------------------------------------------------------------------------------------

% \vfill % Fill the rest of the page with whitespace

\end{titlepage}

\thispagestyle{empty} 
\tableofcontents
\thispagestyle{empty}

\clearpage


\section[Введение]{Введение}

Введение.

\setcounter{page}{1}

\section[Фокус и скалярные аддитивные операторы]{Фокус и скалярные аддитивные операторы}

В данной главе мы опишем наиболее значимые подходы к анализу фокуса и скалярных аддитивных операторов. В чаcти, посвященной семантике фокуса, мы не будем затрагивать проблему фокусного маркирования (см., тем не менее, \citep{selkirk1984phonology,selkirk199516} и  \citep{Schwarzschild1999}). Мы постараемся не ограничиваться генеративной традицией и изложим современные функциональные подходы к анализу фокуса.

\subsection{Семантика фокуса}

\subsection{Семантика фокуса}
Фокус --- грамматическая категория, выделяющая в высказывании информационный компонент, являющийся новым или важным в том смысле, что говорящий не считает его разделенным между собой и слушающим \citep{Jackendoff1972}.

\medskip

Фокус может выражаться при помощи просодических (\ref{pitchAccentF}, фразовое ударение), синтаксических (\ref{cleftF}, клефт) или морфологических средств (\ref{morphemeF}), а также их комбинаций:

\begin{exe}
    \ex
    \begin{xlist}
        \ex \label{pitchAccentF} Я ищу \textbf{Машу}.
        \ex \label{cleftF} It is \textbf{John} we are looking for.
        \ex \label{morphemeF}
            \gll Tí bà wúm-\textbf{á} \textit{kwálíngálá}. \\
                 \textsc{3sg} \textsc{prog} chew-\textsc{foc} colanut \\
            \glt `He is chewing \textsc{colanut}.' \citep[ex.\ 3b]{hartmann}\footnote{\citep{hartmann} замечают, что фокусный показатель действительно присоединяется к предшествующему глаголу.}
    \end{xlist}
\end{exe}

Фокус имеет прямое отношение к семантике высказывания, поскольку способен влиять на его истинностное значение. Ниже приведен классический пример смыслоразличительной функции фокуса из английского языка:

\begin{exe}
    \ex \begin{xlist}
        \ex \label{truthValues1} John only introduced [Bill]\textsubscript{F}\footnote{Здесь и далее под записью []\textsubscript{F} подразумевается интонационное выделение } to Sue.
        \ex \label{truthValues2} John only introduced Bill to [Sue]\textsubscript{F}.
    \end{xlist}
\end{exe}

В примере (\ref{truthValues1}) утверждается, что единственным человеком, которого представили Сью, был Билл. В (\ref{truthValues2}), напротив, говорится, что Билла представили только Сью (в то время как Сью могли также представить Джона, Гарри, etc.)

\medskip

Фокус может быть ассоциирован 

В разделе \ref{alternativeSemantics} мы дадим описание подхода, основанного на понятии \textit{множества альтернатив}.

Пробуем семантику:

\begin{exe}
    \ex $ \llbracket \textsc{only} \rrbracket = \lambda P_{<e,t>}.\lambda x_{e}.[ \forall Q_{<e,t>} [Q(x) \land C(Q)] \rightarrow Q = P ]$
\end{exe}

\subsubsection{Семантика альтернатив (Rooth, 1985)} \label{alternativeSemantics}

Blah.

\bibliography{refs} 
\bibliographystyle{plainnat}

\end{document}