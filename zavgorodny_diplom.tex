\documentclass[a4paper, titlepage, 12pt]{article}
\usepackage[utf8]{inputenc}
\usepackage[russian]{babel}
\usepackage{mathtools}
\usepackage{fixltx2e}
\usepackage{stmaryrd}
\usepackage{gb4e}
\noautomath

\title{Семантика и дистрибуция фокусных частиц в русском языке}

\author{Андрей Завгородний}

\renewcommand*\contentsname{Summary}

\begin{document}

\begin{titlepage}

\newcommand{\HRule}{\rule{\linewidth}{0.5mm}} % Defines a new command for the horizontal lines, change thickness here

\center % Center everything on the page
 
%----------------------------------------------------------------------------------------
%   HEADING SECTIONS
%----------------------------------------------------------------------------------------

\textsc{\large Московский государственный университет имени \\ М.В. Ломоносова}\\[1.5cm] % Name of your university/college
\textsc{\large Филологический факультет}\\[0.5cm] % Major heading such as course name
\textsc{\large Отделение фундаментальной и прикладной лингвистики}\\[0.5cm] % Minor heading such as course title

~\\[2.0cm]

%----------------------------------------------------------------------------------------
%   TITLE SECTION
%----------------------------------------------------------------------------------------


{ \huge Семантика и дистрибуция фокусных частиц в русском языке}\\[0.4cm] % Title of your document
 
%----------------------------------------------------------------------------------------
%   AUTHOR SECTION
%----------------------------------------------------------------------------------------

~\\[3.0cm]

\begin{minipage}{0.4\textwidth}
\begin{flushleft} \large
Дипломная работа студента II курса магистратуры Андрея Олеговича Завгороднего \\
\end{flushleft}
\end{minipage}
~
\begin{minipage}{0.4\textwidth}
\begin{flushright} \large
Научный руководитель д.ф.н., проф. \\  Сергей Георгиевич Татевосов \\
\end{flushright}
\end{minipage}\\[5cm]

% If you don't want a supervisor, uncomment the two lines below and remove the section above
%\Large \emph{Author:}\\
%John \textsc{Smith}\\[3cm] % Your name

%----------------------------------------------------------------------------------------
%   DATE SECTION
%----------------------------------------------------------------------------------------

{\large Москва, 2017} % Date, change the \today to a set date if you want to be precise

%----------------------------------------------------------------------------------------
%   LOGO SECTION
%----------------------------------------------------------------------------------------

%\includegraphics{Logo}\\[1cm] % Include a department/university logo - this will require the graphicx package
 
%----------------------------------------------------------------------------------------

% \vfill % Fill the rest of the page with whitespace

\end{titlepage}

\thispagestyle{empty} 
\tableofcontents
\thispagestyle{empty}

\clearpage


\section[Введение]{Введение}

Введение.

\setcounter{page}{1}

\section[Фокус и скалярные аддитивные операторы]{Фокус и скалярные аддитивные операторы}

В данной главе мы опишем наиболее значимые подходы к анализу фокуса и скалярных аддитивных операторов. В чаcти, посвященной семантике фокуса, мы не будем затрагивать проблему фокусного маркирования (см., тем не менее, Selkirk (1984, 1995) и  Schwarzschild (1999)). Мы постараемся не ограничиваться генеративной традицией и изложим современные функциональные подходы к анализу фокуса.

\subsection[Семантика фокуса]{Семантика фокуса}

Фокус --- грамматическая категория, определяющая ту часть предложения, которая вводит в дискурс новую, не выводимую из предыдущего контекста или контрастивную информацию. Фокус имеет прямое отношение к семантике предложения, поскольку способен влиять на истинностное значение высказывания. Ниже приведен классический пример смыслоразличительной функции фокуса из английского языка:

\begin{exe}
    \ex \begin{xlist}
        \ex \label{ch2ex1} John only introduced [Bill]\textsubscript{F} to Sue.
        \ex \label{ch2ex2} John only introduced Bill to [Sue]\textsubscript{F}.
    \end{xlist}
\end{exe}

В примере (\ref{ch2ex1}) утверждается, что единственным человеком, которого представили Сью, был Билл. В (\ref{ch2ex2}), напротив, говорится, что Билла представили только Сью (в то время как Сью могли также представить Джона, Гарри, etc.)

\medskip

Пробуем семантику:

\begin{exe}
    \ex $ \llbracket \textsc{only} \rrbracket = \lambda P_{<e,t>}.\lambda x_{e}.[ \forall Q_{<e,t>} [Q(x) \land C(Q)] \rightarrow Q = P ]$
\end{exe}

\subsubsection[Семантика альтернатив (Rooth, 1985)]{Семантика альтернатив (Rooth, 1985)}



\newpage
\section{Библиография.}

\begin{itemize}

    
    \item[] \textbf{Rooth, M. 1985.} \textit{Association with Focus.} Ph.D. thesis, UMass. Amherst: Graduate Linguistics Students Association.

    \item[]  \textbf{Rooth, M. 1992.} \textit{A theory of focus interpretation.} Natural Language Semantics 1:75–116. 

    \item[] \textbf{Selkirk, E. 1984.} \textit{Phonology and Syntax: The Relation between Sound and Structure.} Cambridge, MA: MIT Press.

    \item[] \textbf{Selkirk, E. 1995.} \textit{Sentence Prosody: Intonation, Stress, and Phrasing.} In: J. A. Goldsmith (ed.): The Handbook of Phonological Theory. London: Basil Blackwell, pp. 550–569.

    \item[] \textbf{Schwarzschild, R. 1999.} \textit{GIVENness, AvoidF and other Constraints on the Placement of Accent.} Natural Language Semantics 7(2), 141–177.

\end{itemize}


\end{document}